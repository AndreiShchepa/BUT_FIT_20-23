\documentclass[a4paper, 11pt]{article}

\usepackage[czech]{babel}
\usepackage[utf8]{inputenc}
\usepackage[left=2cm, top=3cm, text={17cm, 24cm}]{geometry}
\usepackage{times}
\usepackage[unicode]{hyperref}
\hypersetup{colorlinks = true, hypertexnames = false}

\begin{document}
    \begin{titlepage}
        \begin{center}
            \Huge\textsc{Vysoké učení technické v~Brně} \\
            \huge\textsc{Fakulta informačních technologií} \\
            \vspace{\stretch{0.382}}
            \LARGE{Typografie a publikování\,--\,4. projekt} \\
            \Huge{Bibliografické citace}
            \vspace{\stretch{0.618}}
        \end{center} 
        {\Large \today \hfill Andrei Shchapaniak}
    \end{titlepage}
    
    \section{Úvod}
    
    Jsem si jistý, že každý člověk ve svém životě aspoň jednou potřeboval vytvořit nějaký dokument. Samozřejmě, musíte si vybrat vhodný textový editor, aby práce nad tím dokumentem byla rychlejší a kvalintěší. V 21. století je spousta různých textových editorů, ale dnes se budeme bavit o {\LaTeX}. Proč právě {\LaTeX}? Protože na můj pohled to je nejlepší editor, pokud chcete vytvářet typograficky kvalitní dokumenty. Doporučil bych přečíst porovnání editorů na stránce \cite{oestrem.com}.
    
    \section{Začínáme programovat v~{\LaTeX}u}
    
    Na začátku zavedeme 2 základní definice. 
    \TeX\ je program na sazbu textu. \LaTeX\ je sada maker pro \TeX.
    Dozvědět více o těchto systémech můžete v zajímavé knize \cite{book_2004}.
    Podle názvu sekce jste už mohli pochopit, že práce s {\LaTeX}em je programováni. To je pravda, protože vše sázíte pomocí různých příkazů. Ale pokud neumíte programovat, ještě neznamená, že \LaTeX\ není pro vás. Pro začátečníky bude moc vhodná kniha \cite{book_2011}.
    
    \section{\LaTeX\ pro matematiky}
    
    \LaTeX\ vytváří vzorečky ve vnitřním matematickém módu \$ \ldots \$. Proto sazba matematického textu v {\LaTeX}u je jednodušší, než v jiných textových editorech. Pro mne byla moc zajímavá bakalářská práce k tomuto tématu \cite{master_2}. Taky můžete najít informace na webové stránce \cite{mat_latex}. 
    
    \section{Výhody {\LaTeX}u}
    
    Pokud ještě máte pochybnosti, proč musíte používat právě {\LaTeX}, v následujícím článku najdete důvody \cite{programujte.com}. Samozřejmě, \LaTeX\ není tak dostupný na všech počítačech, jak např. MS Word. Avšak existují online editory pro {\LaTeX}, např. {\LaTeX}lab, Overleaf atd. Více o nich mužete přečíst zde \cite{master_1}
    
    \section{Závěr}
    
    Na závěr bych chtěl tady nechat několik zajímavých článků \cite{article_2013_1} \cite{article_2013_2} \cite{article_2012}. Doufám, že v tomto dokumentu najdete několik užitečných věcí.
    
    \newpage
    \bibliographystyle{czechiso}
	\renewcommand{\refname}{Literatura}
	\bibliography{proj4}

\end{document}
